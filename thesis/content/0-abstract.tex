This paper describes a new automatic method of anger detected from text messages
and reports its accuracy with an experimental evaluation.
For a quick and efficient response in Internet-based customer service, the demand is
increasing for a support system that automatically analyzes customers' messages and
detects customers' anger. Anger is relatively difficult to detect compared with other
emotions in a message because people, especially Japanese people, tend to repress the
expression of strong emotions. Conventional emotion detection systems do not detect
repressed anger with sufficient accuracy.
A new method is proposed to detect customers' anger from text messages. The method
employs a support vector machine with selected features based on studies of irony in
psychology. The features include polysemy, synonymy, polarity of words, and frequency of
appearance in written and spoken languages. The system implementation includes word
properties and frequency dictionaries using several public language resources.
To evaluate the system, a dataset was built from newspaper editorials, because editorials include
opinions, facts, and anger expressions. As a result, the accuracy of the proposed system achieved 41%
whereas a conventional system only achieved 26.0\%.

--------------------------------------------

Over the last 10 years there has been a rapid growth in the usage of social networks. Social net- works like Facebook, Twitter, Instagram or Linkedin have become a daily usage tool for people from different ages and social classes all over the world. People use this social networks to read news, interact with their friends or fellow workers, share their thoughts or concerns, etc. Conse- quently, every hour an enormous amount of public data is generated by users across the globe. The generated information can be applied to several fields like the information retrieval of private organizations or groups. In this research project we are going to analyse how a social bot can infiltrate in an organization and recover information. To this end, a review of the state of the art has been made and an experiment which has proven that the private structure of an organization can be inferred by using centralities and machine learning techniques in an open social network like Twitter.

Keywords
SocialBots, SNA, Machine Learning, Network Theory, Centralities

----------------------------------------------

The growth in the usage of social media, microblog and review platforms has resulted in significant increase of the access to short text messages that reflect individuals opinion and feelings that enable Natural Language Processing. Detecting people's emotions has a wide range of applications such as producing systems that automatically measure the satisfaction of customers and thus can help companies to improve their products or services. This research project focuses on detecting anger, an emotion that is relative difficult to detect compared to other sentiments due to the usage of linguistics figurative language techniques, such as irony, that intends to communicate the opposite of what it is literally said.
To this purpose, a review of the state of the art has been made and an experiment using both traditional machine learning techniques, such as Neural Networks, K-Nearest Neighbor and Support Vector Machines, and Deep learning algorithms has been conducted in an open social network like Twitter.