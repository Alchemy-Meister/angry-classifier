\chapter{Self-assessment}

Personally this project started as a challenge. Throughout my career I have always chosen tasks I have never experienced before as a way to test myself, the work done in this subject has not been an exception. This project has served me as an introduction to research and the scientific world, a field that has been unknown for me until now. In general, I have learned traverse though the references of relevant manuscripts of a research topic, which as enable me to find more interesting documents that develop previous ideas and help me to understand the concepts that with a single paper I would not be able to comprehend.

One of the highest difficulties that I have found when dealing with the investigation topic is to give a formal definition of what repressed anger is and how it manifests. To do so, I had to read thorough previous studies that ended up into psychological documents that explain theories that were totally unrelated with the technical knowledge I was used to work with. 

The worst part by far was to deal with the uncertainty. Even though papers that proved how a architecture performed well with their solution, after spending hundreds of hours searching annotated datasets that could apply to the research topic or implementing a solution that gathers all the data to generate a corpus, design the solution based on combining the ideas extracted from the manuscripts, implementing all parts of the process and manage to evaluate it, there was no way to preview if all the effort would result into scoring good results.

The good this was that with this project, I had the opportunity to dive into artificial, specially into a branch I had almost no knowledge about: machine learning. Parting with a few vague test with Neural Networks and Naive Bayes I started get familiarized with some well-known classifiers while I was reviewing the state of the art, such as Random Forests, \acrshort{knn}, \acrshort{svm}, N-grams based classification, among others, as I studied their mathematical foundations and made empirical testings by using the tools provided by the \acrfull{weka}. In the end, all theses experiments were discarded as possible baseline due to time constrains and focused on the most promising classifier: \acrlong{cnn}, a technique based on the recently popular \acrlong{dl}.

I am glad that I the end I able to base the proposed solution on this thesis on \acrshort{dl}, since is a technology that I was very interested in. This project has enable me to have an introduction to \acrshort{dl} and try to understand the magic that hides beneath and makes it so powerful. It is a precious knowledge that I would like to keep developing in the future.

Moving on to a more technical aspects of the development, this project [TALK ABOUT THE HARDWARE, GPU ACCELERATION, OS INSTALATION, ENVIRONMENT SETTING, DEBUGING AND TESTING, LONG EXECUTION TIMES, ETC. THAT HAVE MADE ME REMEMBER SOME CONTENT STUDIED DURING MY BACHELLOR.]

[CONCLUDE WITH THE DESIRE TO CONTINUE WITH RESEARH, AS I HAVE TO WORK ON THE PDH PROPOSAL]

\iffalse
However, I have also learnt how difficult it is to find a research field where you feel comfortable with and able to find new research opportunities where you can contribute to. In addition, even though it did not happen to me, I witnessed how a person worked for a couple months and in the end do not achieve the desired results. In fact, this was my main concern during this master’s thesis. I spent around 100 hours manually annotating each organization’s employees’ position inside the company, without being certain if I was going to achieve good enough results or not.
Furthermore, I have also had the chance to delve into artificial intelligence and realize that AI is a field that I would enjoy to keep working on. In addition, I have also had the opportunity to present this research in Databeers Bilbao with a very good acceptance, which I am very pleased about.
All in all, this project has helped me to understand in my own skin, how different research is from my previous experiences, which were more technical. Even though it has been quite a challenge in comparison to what I have been doing until now, ultimately resulted in a rewarding experience, and by being so, I would be thrilled to continue doing research and prove what I am capable of.

\fi