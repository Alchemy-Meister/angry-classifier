\chapter{Self-assessment}

Personally this project started as a challenge. Throughout my career I have always chosen tasks I have never experienced before as a way to test myself, the work done in this subject has not been an exception. This project has served me as an introduction to research and the scientific world, a field that has been unknown for me until now. In general, I have learned traverse though the references of relevant manuscripts of a research topic, which as enable me to find more interesting documents that develop previous ideas and help me to understand the concepts that with a single paper I would not be able to comprehend.

One of the highest difficulties that I have found when dealing with the investigation topic is to give a formal definition of what repressed anger is and how it manifests. To do so, I had to read thorough previous studies that ended up into psychological documents that explain theories that were totally unrelated with the technical knowledge I was used to work with. 

The worst part by far was to deal with the uncertainty. Even though papers that proved how a architecture performed well with their solution, after spending hundreds of hours searching annotated datasets that could apply to the research topic or implementing a solution that gathers all the data to generate a corpus, design the solution based on combining the ideas extracted from the manuscripts, implementing all parts of the process and manage to evaluate it, there was no way to preview if all the effort would result into scoring good results.

The good this was that with this project, I had the opportunity to dive into artificial, specially into a branch I had almost no knowledge about: machine learning. Parting with a few vague test with Neural Networks and Naive Bayes I started get familiarized with some well-known classifiers while I was reviewing the state of the art, such as Random Forests, \acrshort{knn}, \acrshort{svm}, N-grams based classification, among others, as I studied their mathematical foundations and made empirical testings by using the tools provided by the \acrfull{weka}. In the end, all theses experiments were discarded as possible baseline due to time constrains and focused on the most promising classifier: \acrlong{cnn}, a technique based on the recently popular \acrlong{dl}.

I am glad that I the end I able to base the proposed solution on this thesis on \acrshort{dl}, since is a technology that I was very interested in. This project has enable me to have an introduction to \acrshort{dl} and try to understand the magic that hides beneath and makes it so powerful. It is a precious knowledge that I would like to keep developing in the future.

Moving on to a more technical aspects of the development, this project has enable me to review and potential some skills that a learned during my bachelor's degree. As execution time of machine learning related task tend to be long, the need of optimize the code, debugging and to develop small unit-testing modules to ensure that the execution will not end up into an error was required before relying into investment of high computational horsepower equipment. To justify the need of new hardware that would speed up the execution of the learning process and thus, ensure a constant progression of the project during the given time, I had to invest time on researching performance gain from \acrshort{cpu} only to \acrshort{gpgpu} enabled systems on deep learning tasks, for which having a own \acrshort{gpgpu} home-server and the knowledge on how to set a \acrshort{vpn} enable to establish connection to it and run benchmarks against main laboratory's Xeon powered server helped to show empirical results of the performance gain.

I am proud to have set a fully working \acrshort{gpgpu} system from the recently acquired equipment, able to development of \acrshort{dl} projects based on keras \cite{keras} on both Tensorflow and Theano back-ends and thus, run CUDA tasks with optional \acrshort{cudnn} accelerator \cite{chetlur2014cudnn} along to a installation manual that hopefully will help future research students to avoid the errors and complications I have been through as my contribution to the \acrfull{ksl}.

Finally, I would like to communicate my desire to continue working on research field by undergoing a Ph.D course after I graduate from the current master course and thus, keep potentiating my skills in artificial intelligence and produce and evidence to contribution to knowledge for the scientific community and to society in general.
