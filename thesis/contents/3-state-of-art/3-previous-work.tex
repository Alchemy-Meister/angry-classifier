% !TeX root = repressed-anger.tex

\section{Previous Work}
\label{sec:previous_work}

The study of anger and its management has been studied since ancient times. According to \acrfull{apa}, anger is ``an emotion characterized by antagonism toward someone or something you feel has deliberately done you wrong'' \cite{angerAPA}. Anger is a normal emotion all human experience. Modern psychologists have studied the health effects that may derivate from anger suppression, such as shown in \cite{kemp1995anger} and \cite{staicu2010anger}, linking it to a more extremist and sarcastic personality. This research aims to find the answer to the question: ``Can the repressed anger be detected in text?''. Based on the definition of personality of a subject that has experienced repressed anger, the problem formulation has been simplified into finding negative messages influenced by anger that covered by a layer of sarcasm. The following subsections present a brief summary of the research conducted in the detection of anger and sarcasm.

\subsection{Anger}
\label{subsec:anger}

TODO

\subsection{Irony and Sarcasm}
\label{subsec:irony_sarcasm}



Defend the idea of using irony and sarcasm as the same, since sarcasm is a subset of irony.
``In linguistics, verbal irony is sometimes used as a synonym of sarcasm \cite{giora2013negation}; \cite{giora2015defaultness} and \cite{giora2015default}. According to the literature, boundaries in meaning between irony, sarcasm et similia are fuzzy \cite{bosco2013developing}. While some authors consider irony as an umbrella term covering also sarcasm \cite{gibbs1991psychological}, \cite{wilson2006pragmatics} and \cite{kreuz1993empirical}, others provide insights for a separation \cite{filatova2012irony}.''


Survey of the approaches used to detect sarcasm. Three approaches, rule-based, statistical approaches, learning approaches. Inside Learning approaches there are architectures based on deep learning \cite{joshi2016automatic}.

Detection of Portuguese irony using pattern matching rules such as diminutive forms, interjections, verb morphology, heavy punctuation, quotation marks, laughter expressions, among others. Classifying the sentences as ironic, not ironic, undecided and ambiguous. They concluded stating that the best patters for irony recognition are quote and laugh obtaining a precision of 68.29\% and 85.40\% respectively \cite{carvalho2009clues}.

Semi-supervised techniques for sarcastic sentences in Twitter and Amazon, 5.9 million unique tweets, use of \#sarcasm hashtag, removal of link, users and hashtags, pattern extraction, such as \acrfullpl{hfw}, \acrfullpl{cw}, punctuation based features to create single entry feature vectors. classification range of 1 to 5 to measure the presence of sarcasm, with a \acrfull{knn}-like strategy \cite{davidov2010enhanced}. Evaluation using 5-fold cross validation using all the features obtained the highest F-score of 0.545 on Twitter and 0.827 in Amazon \cite{davidov2010semi}.