\chapter*{Abstract}

The growth in the usage of social media, microblog and review platforms has resulted in a significant increase of the access to short text messages that reflect individual's opinion and feelings. Automatic detection of people's emotions has a wide range of applications such as producing systems that measure the satisfaction of customers and thus help companies to improve their products or services. This research project focuses on detecting anger, an emotion that is relative difficult to detect compared to other sentiments due to the usage of linguistics figurative language techniques, such as irony, that intends to communicate the opposite of what it is literally said.
To this purpose, a review of the state of the art has been made and an experiment using not only traditional machine learning techniques, such as Neural Networks, K-Nearest Neighbor and Support Vector Machines, but also Deep learning algorithms has been conducted in an open social network like Twitter.
The proposed methods define the repressed anger detection as a classification problem and to solve it, the task is divided into two subtasks that complement each other. The first focuses on explicit anger detection, while the second subtask's goal is to detect irony. The system make use of selected featured based on characteristics properties of English language and studies emotion and irony in psychology. The model is composed on features such as: frequency of words, style in written and spoken languages, intensity of adverbs and adjectives, structure of the document, use of emoticons, synonymy, ambiguity and the contrast of sentiment and negative situation.

\vspace{2em}

{\Large\bfseries\sffamily Keywords}
\vspace{3\medskipamount}

Emotion Analysis, Supervised Classification, One-vs-all Classification, Convolutional Neural Networks, Repressed Anger Detection